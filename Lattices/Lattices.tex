%% This file is here to make it convenient to view various diagrams in this directory.
\documentclass{amsart}
\usepackage{amsmath}%
\usepackage{amsfonts}%
\usepackage{amssymb}%
\usepackage{mathrsfs}
\usepackage{stmaryrd,enumerate}
\usepackage{graphicx}
\usepackage{comment}
\usepackage{tikz}
\usepackage{xspace}

\begin{document}

%%///////////////////////////////////////////////////////////////////////////////////
%% tikz notes:
%% For Hasse diagrams, instead of hard-coding the size of the circle representing
%% points in a lattice, please use \dotsize.  For example,
%%     \node (bottom) at (0,0) [draw, circle,inner sep=\dotsize] {};
%% or
%%     \node[lat] (bottom) at (0,0) {};  (see comment on the "lat" style below)
\newcommand{\dotsize}{0.8pt}
%%
%% To create nodes of lattices in a uniform and consistent way, we define
\tikzstyle{lat} = [circle,draw,inner sep=\dotsize]
%% To put a lattice node named "mynode" at the point (x,y)=(1,2) in the figure,
%% put the following inside your tikzpicture block:
%%     \node[lat] (mynode) at (1,2) {};

% To scale all diagrams uniformly, change this setting:
\newcommand{\figscale}{1}


  \begin{tikzpicture}[scale=\figscale]
    \input{M4.tex}
    \draw (0,0) node {$(0,0)$};
    \draw (0,-1) node {{\tt file: M4.tex}};
  \end{tikzpicture}

  \begin{tikzpicture}[scale=\figscale]
    \input{M4-big.tex}
    \draw (0,0) node {$(0,0)$};
    \draw (0,-1) node {{\tt file: M4-big.tex}};
  \end{tikzpicture}

  \begin{tikzpicture}[scale=\figscale]
    \input{L3.tex}
    \draw (0,0) node {$(0,0)$};
    \draw (0,-1) node {{\tt file: L3.tex}};
  \end{tikzpicture}

  \begin{tikzpicture}[scale=\figscale]
    \input{L3-big.tex}
    \draw (0,0) node {$(0,0)$};
    \draw (0,-1) node {{\tt file: L3-big.tex}};
  \end{tikzpicture}

  \begin{tikzpicture}[scale=\figscale]
    %%  The elusive winged-2x3 %%
\node[lat] (bottom) at (0,0) {};
\node[lat] (top) at (1,3) {};
\node[lat] (n11) at (-1,1) {};
\node[lat] (11) at (1,1)  {};
\node[lat] (02) at (0,2)  {};
\node[lat] (22) at (2,2)  {};
\node[lat] (wing) at (-2,1) {};
\draw[semithick] 
(bottom) to (11) to (22) to (top) to (02) to (11)
(top) to (wing) to (bottom) to (n11) to (02);

%% the old way (too complicated):
%% \node[lat] (01) at (0,1)  [draw, circle, inner sep=\dotsize] {};
%% \foreach \j in {0,2} 
%%          { \node[lat] (1\j) at (1,\j) {};}
%%          \foreach \j in {1,3} 
%%                   { \node[lat] (2\j) at (2,\j) {};}
%%                   { \node[lat] (32) at (3,2) {};}

%% \draw[semithick] (10) to (01) to (12) to (23) to (32) to (21) to (10) (21) to (12);
%% { \node[lat] (m11) at (-1,1) {};}
%% \draw[semithick] (10) to (m11) to (23);


    \draw (0,0) node {$(0,0)$};
    \draw (0,-1) node {{\tt file: L10.tex}};
  \end{tikzpicture}

  \begin{tikzpicture}[scale=\figscale]
    \input{L13.tex}
    \draw (0,0) node {$(0,0)$};
    \draw (0,-1) node {{\tt file: L13.tex}};
  \end{tikzpicture}

  \begin{tikzpicture}[scale=\figscale]
    \input{L17.tex}
    \draw (0,0) node {$(0,0)$};
    \draw (0,-1) node {{\tt file: L17.tex}};
  \end{tikzpicture}

  \begin{tikzpicture}[scale=\figscale]
    \input{L19.tex}
    \draw (0,0) node {$(0,0)$};
    \draw (0,-1) node {{\tt file: L19.tex}};
  \end{tikzpicture}

  \begin{tikzpicture}[scale=\figscale]
    \input{L20.tex}
    \draw (0,0) node {$(0,0)$};
    \draw (0,-1) node {{\tt file: L20.tex}};
  \end{tikzpicture}

  \begin{tikzpicture}[scale=\figscale]
    
      \node (x) at (-1,1)  [draw, circle, inner sep=\dotsize] {};
      \node (y) at (1,1)  [draw, circle, inner sep=\dotsize] {};
      \node (a) at (0,2)  [draw, circle, inner sep=\dotsize] {};
      \node (b) at (0,0)  [draw, circle, inner sep=\dotsize] {};
      \draw (x) node [left] {$x$};
      \draw (a) node [above] {$a$};
      \draw (b) node [below] {$b$};
      \draw (y) node [right] {$y$};
      \draw[semithick] (b)-- (a) node[pos=.5,right] {$\beta$};
      \draw[semithick] (x)-- (a) node[pos=.5,left] {$\alpha_1$};
      \draw[semithick] (x)-- (b) node[pos=.5,left] {$\alpha_2$};
      \draw[semithick] (b)-- (y) node[pos=.5,right] {$\alpha_1$};
      \draw[semithick] (a)-- (y) node[pos=.5,right] {$\alpha_2$};
%%   \caption{The Wheatstone Bridge which defines the relation 
%%     $\tau(\alpha_1, \alpha_2, \beta)$ as follows: 
%%     $(x,y) \in \tau(\alpha_1, \alpha_2, \beta)$ if and only if there 
%%     exist $a, b \in X$ satisfying the relations in the diagram.}
%%   \label{fig:rhoagain}
%% \end{figure}

    \draw (0,0) node {$(0,0)$};
    \draw (0,-1) node {{\tt file: N54W.tex}};
  \end{tikzpicture}

  \begin{tikzpicture}[scale=\figscale]
    \input{wheatstone_bridge.tex}
    \draw (0,0) node {$(0,0)$};
    \draw (0,-1) node {{\tt file: weatstone\_bridge.tex}};
  \end{tikzpicture}


\end{document}
