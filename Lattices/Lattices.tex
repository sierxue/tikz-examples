%% This file is here to make it convenient to view various diagrams in this directory.
\documentclass{amsart}
\usepackage{amsmath}%
\usepackage{amsfonts}%
\usepackage{amssymb}%
\usepackage{mathrsfs}
\usepackage{stmaryrd,enumerate}
\usepackage{graphicx}
\usepackage{comment}
\usepackage{tikz}
\usepackage{xspace}

\begin{document}

%%///////////////////////////////////////////////////////////////////////////////////
%% tikz notes:
%% For Hasse diagrams, instead of hard-coding the size of the circle representing
%% points in a lattice, please use \dotsize.  For example,
%%     \node (bottom) at (0,0) [draw, circle,inner sep=\dotsize] {};
%% or
%%     \node[lat] (bottom) at (0,0) {};  (see comment on the "lat" style below)
\newcommand{\dotsize}{0.8pt}
%%
%% To create nodes of lattices in a uniform and consistent way, we define
\tikzstyle{lat} = [circle,draw,inner sep=\dotsize]
%% To put a lattice node named "mynode" at the point (x,y)=(1,2) in the figure,
%% put the following inside your tikzpicture block:
%%     \node[lat] (mynode) at (1,2) {};

% To scale all diagrams uniformly, change this setting:
\newcommand{\figscale}{1}


  \begin{tikzpicture}[scale=\figscale]
    %% File: M4.tex
%% Author: williamdemeo@gmail.com
%% Copyright (C) 2015 William DeMeo

\node[lat] (bot) at (0,0) {};
\node[lat] (top) at (0,2) {};
\node[lat] (a) at (-1,1) {};
\node[lat] (b) at (-.5,1) {};
\node[lat] (c) at (.5,1) {};
\node[lat] (d) at (1,1) {};
\draw[semithick] (bot) -- (a) -- (top) -- (b) -- (bot) -- (c) -- (top) -- (d) -- (bot);

        %% \node (250) at (2.5,0)  [draw, circle, inner sep=\dotsize] {};
        %% \node (253) at (2.5,3)  [draw, circle, inner sep=\dotsize] {};
        %% \foreach \i in {1,...,4} 
        %%          { 
        %%            \node (\i15) at (\i,1.5)  [draw, circle, inner sep=\dotsize] {};
        %%            \draw[semithick] (250) to (\i15) to (253);
        %%          }

    \draw (0,0) node {$(0,0)$};
    \draw (0,-1) node {{\tt file: M4.tex}};
  \end{tikzpicture}

  \begin{tikzpicture}[scale=\figscale]
    %% M4-big.tex
\node[lat] (bot) at (0,0) {};
\node[lat] (top) at (0,3) {};
\node[lat] (a) at (-1.5,1.5) {};
\node[lat] (b) at (-.75,1.5) {};
\node[lat] (c) at (.75,1.5) {};
\node[lat] (d) at (1.5,1.5) {};
\draw[semithick] (bot) -- (a) -- (top) -- (b) -- (bot) -- (c) -- (top) -- (d) -- (bot);

%% the old way (too complicated):
        %% \node (250) at (2.5,0)  [draw, circle, inner sep=\dotsize] {};
        %% \node (253) at (2.5,3)  [draw, circle, inner sep=\dotsize] {};
        %% \foreach \i in {1,...,4} 
        %%          { 
        %%            \node (\i15) at (\i,1.5)  [draw, circle, inner sep=\dotsize] {};
        %%            \draw[semithick] (250) to (\i15) to (253);
        %%          }

    \draw (0,0) node {$(0,0)$};
    \draw (0,-1) node {{\tt file: M4-big.tex}};
  \end{tikzpicture}

  \begin{tikzpicture}[scale=\figscale]
    % L11 aka L3 
        \node (bottom) at (0,0)  [draw, circle, inner sep=\dotsize] {};
        \node (top) at (2,6)  [draw, circle, inner sep=\dotsize] {};
        \node (n22) at (-2,2)  [draw, circle, inner sep=\dotsize] {};
        \node (22) at (2,2)  [draw, circle, inner sep=\dotsize] {};
        \node (13) at (1,3)  [draw, circle, inner sep=\dotsize] {};
        \node (04) at (0,4)  [draw, circle, inner sep=\dotsize] {};
        \node (44) at (4,4)  [draw, circle, inner sep=\dotsize] {};

        \draw[semithick] (bottom) to (22) to (44) to (top) to (04) to (13) to (22)
        (bottom) to (n22) to (04);

    \draw (0,0) node {$(0,0)$};
    \draw (0,-1) node {{\tt file: L3.tex}};
  \end{tikzpicture}

  \begin{tikzpicture}[scale=\figscale]
    %% File: L3-big.tex
%% Author: williamdemeo@gmail.com
%% Copyright (C) 2015 William DeMeo

% L11 aka L3 
\node[lat] (bottom) at (0,0) {};
\node[lat] (top) at (2,6) {};
\node[lat] (n22) at (-2,2) {};
\node[lat] (22) at (2,2)  {};
\node[lat] (13) at (1,3)  {};
\node[lat] (04) at (0,4)  {};
\node[lat] (44) at (4,4)  {};

\draw[semithick] (bottom) to (22) to (44) to (top) to (04) to (13) to (22)
(bottom) to (n22) to (04);

    \draw (0,0) node {$(0,0)$};
    \draw (0,-1) node {{\tt file: L3-big.tex}};
  \end{tikzpicture}

  \begin{tikzpicture}[scale=\figscale]
    %% File: L10.tex
%% Author: williamdemeo@gmail.com
%% Copyright (C) 2015 William DeMeo

%%  This is the elusive winged-2x3. At some point we started 
%%  referring to this lattice as L7 because it's the only 
%%  seven element lattice for which don't have a representation
%%  (as a congruence lattice of a finite algebra).

\node[lat] (bottom) at (0,0) {};
\node[lat] (top) at (1,3) {};
\node[lat] (n11) at (-1,1) {};
\node[lat] (11) at (1,1)  {};
\node[lat] (02) at (0,2)  {};
\node[lat] (22) at (2,2)  {};
\node[lat] (wing) at (-2,1) {};
\draw[semithick] 
(bottom) to (11) to (22) to (top) to (02) to (11)
(top) to (wing) to (bottom) to (n11) to (02);


    \draw (0,0) node {$(0,0)$};
    \draw (0,-1) node {{\tt file: L10.tex}};
  \end{tikzpicture}

  \begin{tikzpicture}[scale=\figscale]
    %\newcommand{\dotsize}{1};
\node[lat] (0) at (0,0) {};
\node[lat] (1) at (-0,1) {};
\node[lat] (2) at (1,2) {};
\node[lat] (3) at (-1,2) {};
\node[lat] (4) at (-0,2) {};
\node[lat] (5) at (-0,3) {};
\node[lat] (6) at (-0,4) {};
%\draw[font=\scriptsize] (0,-.5) node {[A4, (C2 x C2 x C2 x C2) : A5]};
%\draw[font=\scriptsize] (0,-1) node {SmallGroup(960,11358) Index 80};

\draw[semithick]
(0) to (1) to (4) to (5) to (6)
(0) to (2) to (6) to (3) to (0);


    \draw (0,0) node {$(0,0)$};
    \draw (0,-1) node {{\tt file: L13.tex}};
  \end{tikzpicture}

  \begin{tikzpicture}[scale=\figscale]
    
        \node (bottom) at (0,0)  [draw, circle, inner sep=\dotsize] {};
        \node (top) at (0,5)  [draw, circle, inner sep=\dotsize] {};
        \node (n115) at (-1,1.5)  [draw, circle, inner sep=\dotsize] {};
        \node (015) at (0,1.5)  [draw, circle, inner sep=\dotsize] {};
        \node (115) at (1,1.5)  [draw, circle, inner sep=\dotsize] {};
        \node (n252) at (-2.5,2)  [draw, circle, inner sep=\dotsize] {};
        \node (03) at (0,3)  [draw, circle, inner sep=\dotsize] {};
        \draw[semithick] 
        (bottom) to (n252) to (top)
        (bottom) to (n115) to (03) to (115) to
        (bottom) to (015) to (03) to (top);
       %\draw (-4.5,2.2) node {$L \cong$};



    \draw (0,0) node {$(0,0)$};
    \draw (0,-1) node {{\tt file: L17.tex}};
  \end{tikzpicture}

  \begin{tikzpicture}[scale=\figscale]
    % L19
        \node (bottom) at (0,0)  [draw, circle, inner sep=\dotsize] {};
        \node (top) at (0,3)  [draw, circle, inner sep=\dotsize] {};
        \node (11) at (1,1)  [draw, circle, inner sep=\dotsize] {};
        \node (n11) at (-1,1)  [draw, circle, inner sep=\dotsize] {};
        \node (12) at (1,2)  [draw, circle, inner sep=\dotsize] {};
        \node (n12) at (-1,2)  [draw, circle, inner sep=\dotsize] {};
        \node (01) at (0,1)  [draw, circle, inner sep=\dotsize] {};

        \draw[semithick] (bottom) to (11) to (12) to (top) to (n12) to (n11) to (bottom) to (01) to (12);

    \draw (0,0) node {$(0,0)$};
    \draw (0,-1) node {{\tt file: L19.tex}};
  \end{tikzpicture}

  \begin{tikzpicture}[scale=\figscale]
    % L20
      \foreach \j in {0,...,4}
      {
        \node (0\j) at (0,\j)  [draw, circle, inner sep=\dotsize] {};
      }
      \node (152) at (1.5,2)  [draw, circle, inner sep=\dotsize] {};
      \node (n125) at (-1,2.5)  [draw, circle, inner sep=\dotsize] {};
      \draw[semithick] (01) to (02) to (03) to (04) to (n125) to (01) to (00) to (152) to (04);

    \draw (0,0) node {$(0,0)$};
    \draw (0,-1) node {{\tt file: L20.tex}};
  \end{tikzpicture}

  \begin{tikzpicture}[scale=\figscale]
    
      \node (x) at (-1,1)  [draw, circle, inner sep=\dotsize] {};
      \node (y) at (1,1)  [draw, circle, inner sep=\dotsize] {};
      \node (a) at (0,2)  [draw, circle, inner sep=\dotsize] {};
      \node (b) at (0,0)  [draw, circle, inner sep=\dotsize] {};
      \draw (x) node [left] {$x$};
      \draw (a) node [above] {$a$};
      \draw (b) node [below] {$b$};
      \draw (y) node [right] {$y$};
      \draw[semithick] (b)-- (a) node[pos=.5,right] {$\beta$};
      \draw[semithick] (x)-- (a) node[pos=.5,left] {$\alpha_1$};
      \draw[semithick] (x)-- (b) node[pos=.5,left] {$\alpha_2$};
      \draw[semithick] (b)-- (y) node[pos=.5,right] {$\alpha_1$};
      \draw[semithick] (a)-- (y) node[pos=.5,right] {$\alpha_2$};
%%   \caption{The Wheatstone Bridge which defines the relation 
%%     $\tau(\alpha_1, \alpha_2, \beta)$ as follows: 
%%     $(x,y) \in \tau(\alpha_1, \alpha_2, \beta)$ if and only if there 
%%     exist $a, b \in X$ satisfying the relations in the diagram.}
%%   \label{fig:rhoagain}
%% \end{figure}

    \draw (0,0) node {$(0,0)$};
    \draw (0,-1) node {{\tt file: N54W.tex}};
  \end{tikzpicture}

  \begin{tikzpicture}[scale=\figscale]
    %% wheatstone_bridge.tex
%% "Wheatstone Bridge"
\node[lat] (x) at (-1,1) {};
\node[lat] (y) at (1,1) {};
\node[lat] (a) at (0,2) {};
\node[lat] (b) at (0,0) {};
\draw (x) node [left] {$x$};
\draw (a) node [above] {$a$};
\draw (b) node [below] {$b$};
\draw (y) node [right] {$y$};
\draw[semithick] (b)-- (a) node[pos=.5,right] {$\beta$};
\draw[semithick] (x)-- (a) node[pos=.5,left] {$\alpha_1$};
\draw[semithick] (x)-- (b) node[pos=.5,left] {$\alpha_2$};
\draw[semithick] (b)-- (y) node[pos=.5,right] {$\alpha_1$};
\draw[semithick] (a)-- (y) node[pos=.5,right] {$\alpha_2$};
%%   \caption{The Wheatstone Bridge which defines the relation 
%%     $\tau(\alpha_1, \alpha_2, \beta)$ as follows: 
%%     $(x,y) \in \tau(\alpha_1, \alpha_2, \beta)$ if and only if there 
%%     exist $a, b \in X$ satisfying the relations in the diagram.}


    \draw (0,0) node {$(0,0)$};
    \draw (0,-1) node {{\tt file: weatstone\_bridge.tex}};
  \end{tikzpicture}


\end{document}
