\documentclass{amsart}
\usepackage{tikz,calc,scalefnt,xspace,url}

\begin{document}

\noindent {\Large Tikz examples for depicting overalgebras or ``rabbit ears.''}

\bigskip

%%%%%%%%%%%%%%%%%%%%%%%%%%%%%%%%%%%%%%%%%%%%%%%%%%%%%%%%%%%%%%%%%%%%%%%%%%
{\it Example 1:} 
    {\bf Lines} (resp., dotted lines) surround {\bf sets} (resp., blocks).

    \tikzstyle{blocks} = [rounded corners,dotted,semithick]
    \tikzstyle{sets} = [rounded corners]

    %% (so we remember to adjust the fig caption if necessary--see Example 2 below)
    \newcommand{\mycaption}{Dotted lines delineate the congruence classes
      of $\beta^\star$ (left) and $\tbeta$ (right); solid lines surround the sets $B_i$.}

    \newcommand{\tbeta}{\ensuremath{\widetilde{\beta}}}
    \newcommand{\ttheta}{\ensuremath{\widetilde{\theta}}}
    \newcommand{\hbeta}{\ensuremath{\widehat{\beta}}}


\begin{figure}[h]
  \centering
      {\scalefont{.8}
        \begin{tikzpicture}[scale=.7]

          % lines around the minimal universes (the $B_i$'s)
          \draw[sets]  
          (-1.5,-1.5) rectangle (1.5,1.5) % Center
          (.5,.5) rectangle (3.5,3.5)     % NE
          (.5,-3.5) rectangle (3.5,-.5)   % SE
          (-3.5,-3.5) rectangle (-.5,-.5) % SW
          (-3.5,.5) rectangle (-.5,3.5);  % NW


          % lines around equivalence classes 
          \draw[blocks] 
          (-1.35,-.35) rectangle (1.35,.35) % Center
          (-3.35,.65) rectangle (3.35,1.35)   % b1(0:2) b2(0:2)
          (-3.35,-.65) rectangle (3.35,-1.35) % b3(6:8) b4(6:8)
          (-3.35,1.65) rectangle (-.65,2.35)
          (-3.35,2.65) rectangle (-.65,3.35)
          (-3.35,-1.65) rectangle (-.65,-2.35)
          (-3.35,-2.65) rectangle (-.65,-3.35)
          (4-3.35,1.65) rectangle (4-.65,2.35)
          (4-3.35,2.65) rectangle (4-.65,3.35)
          (4-3.35,-1.65) rectangle (4-.65,-2.35)
          (4-3.35,-2.65) rectangle (4-.65,-3.35);


          % element labels (the $b^i_j$'s)
          \draw 
          %%% B1
          (-3,3) node {$b^1_8$}   (-2,3) node {$b^1_7$}   (-1,3) node {$b^1_6$}
          (-3,2) node {$b^1_5$}   (-2,2) node {$b^1_4$}   (-1,2) node {$b^1_3$}
          (-3,1) node {$b^1_2$}   (-2,1) node {$b^1_1$}   (-1,1) node {$b_0$}

          ( 0,1) node {$b_1$}
          
          %%% B2
          (1,3) node {$b^2_8$}  (2,3) node {$b^2_7$}   (3,3) node {$b^2_6$}
          (1,2) node {$b^2_5$}  (2,2) node {$b^2_4$}   (3,2) node {$b^2_3$}
          (1,1) node {$b_2$}    (2,1) node {$b^2_1$}   (3,1) node {$b^2_0$}

          %%% center block
          (-1,0) node {$b_3$}   ( 0,0) node {$b_4$}    ( 1,0) node {$b_5$}
          
          %%% B3
          (-3,-1) node {$b^3_8$}  (-2,-1) node {$b^3_7$}  (-1,-1) node {$b_6$}
          (-3,-2) node {$b^3_5$}  (-2,-2) node {$b^3_4$}  (-1,-2) node {$b^3_3$}
          (-3,-3) node {$b^3_2$}  (-2,-3) node {$b^3_1$}  (-1,-3) node {$b^3_0$}

          (0,-1) node {$b_7$}

          %%% B4
          (1,-1) node {$b_8$}   (2,-1) node {$b^4_7$}   (3,-1) node {$b^4_6$}
          (1,-3) node {$b^4_2$} (2,-3) node {$b^4_1$}   (3,-3) node {$b^4_0$}
          (1,-2) node {$b^4_5$} (2,-2) node {$b^4_4$}   (3,-2) node {$b^4_3$};


          \draw (0, -4) node {$\beta^\star$ $(=\beta^*)$};
        \end{tikzpicture}
        \hskip1cm
        \begin{tikzpicture}[scale=.7]
          % lines around the minimal universes (the $B_i$'s)
          \draw[sets]  
          (-1.5,-1.5) rectangle (1.5,1.5) % Center
          (.5,.5) rectangle (3.5,3.5)     % NE
          (.5,-3.5) rectangle (3.5,-.5)   % SE
          (-3.5,-3.5) rectangle (-.5,-.5) % SW
          (-3.5,.5) rectangle (-.5,3.5);  % NW


          % lines around equivalence classes 
          \draw[blocks] 
          (-1.35,-.35) rectangle (1.35,.35) % Center
          (-3.35,.65) rectangle (3.35,1.35)   % b1(0:2) b2(0:2)
          (-3.35,-.65) rectangle (3.35,-1.35) % b3(6:8) b4(6:8)
          (-3.35,1.65) rectangle (4-.65,2.35)
          (-3.35,2.65) rectangle (4-.65,3.35)
          (-3.35,-1.65) rectangle (4-.65,-2.35)
          (-3.35,-2.65) rectangle (4-.65,-3.35);


          % element labels (the $b^i_j$'s)
          \draw 
          %%% B1
          (-3,3) node {$b^1_8$}   (-2,3) node {$b^1_7$}   (-1,3) node {$b^1_6$}
          (-3,2) node {$b^1_5$}   (-2,2) node {$b^1_4$}   (-1,2) node {$b^1_3$}
          (-3,1) node {$b^1_2$}   (-2,1) node {$b^1_1$}   (-1,1) node {$b_0$}

          ( 0,1) node {$b_1$}
          
          %%% B2
          (1,3) node {$b^2_8$}  (2,3) node {$b^2_7$}   (3,3) node {$b^2_6$}
          (1,2) node {$b^2_5$}  (2,2) node {$b^2_4$}   (3,2) node {$b^2_3$}
          (1,1) node {$b_2$}    (2,1) node {$b^2_1$}   (3,1) node {$b^2_0$}

          %%% center block
          (-1,0) node {$b_3$}   ( 0,0) node {$b_4$}    ( 1,0) node {$b_5$}
          
          %%% B3
          (-3,-1) node {$b^3_8$}  (-2,-1) node {$b^3_7$}  (-1,-1) node {$b_6$}
          (-3,-2) node {$b^3_5$}  (-2,-2) node {$b^3_4$}  (-1,-2) node {$b^3_3$}
          (-3,-3) node {$b^3_2$}  (-2,-3) node {$b^3_1$}  (-1,-3) node {$b^3_0$}

          (0,-1) node {$b_7$}

          %%% B4
          (1,-1) node {$b_8$}   (2,-1) node {$b^4_7$}   (3,-1) node {$b^4_6$}
          (1,-3) node {$b^4_2$} (2,-3) node {$b^4_1$}   (3,-3) node {$b^4_0$}
          (1,-2) node {$b^4_5$} (2,-2) node {$b^4_4$}   (3,-2) node {$b^4_3$};


          \draw (0, -4) node {$\tbeta$ $(=\hbeta)$};
        \end{tikzpicture}
      }
      \caption{\mycaption}
      %% \label{fig:overalgebra1}
\end{figure}


%%%%%%%%%%%%%%%%%%%%%%%%%%%%%%%%%%%%%%%%%%%%%%%%%%%%%%%%%%%%%%%%%%%%%%%%%%
{\it Example 2:} 
    {\bf Lines} (resp., dotted lines) surround {\bf blocks} (resp., sets).

    \tikzstyle{blocks} = [rounded corners]
    \tikzstyle{sets} = [rounded corners,dotted,semithick]

    %% adjust the figure caption:
    \renewcommand{\mycaption}{Solid lines delineate the congruence classes
      of $\beta^\star$ (left) and $\tbeta$ (right); dotted lines surround the sets $B_i$.}

\begin{figure}[h]
  \centering
      {\scalefont{.8}
        \begin{tikzpicture}[scale=.7]

          % lines around the minimal universes (the $B_i$'s)
          \draw[sets]  
          (-1.5,-1.5) rectangle (1.5,1.5) % Center
          (.5,.5) rectangle (3.5,3.5)     % NE
          (.5,-3.5) rectangle (3.5,-.5)   % SE
          (-3.5,-3.5) rectangle (-.5,-.5) % SW
          (-3.5,.5) rectangle (-.5,3.5);  % NW


          % lines around equivalence classes 
          \draw[blocks] 
          (-1.35,-.35) rectangle (1.35,.35) % Center
          (-3.35,.65) rectangle (3.35,1.35)   % b1(0:2) b2(0:2)
          (-3.35,-.65) rectangle (3.35,-1.35) % b3(6:8) b4(6:8)
          (-3.35,1.65) rectangle (-.65,2.35)
          (-3.35,2.65) rectangle (-.65,3.35)
          (-3.35,-1.65) rectangle (-.65,-2.35)
          (-3.35,-2.65) rectangle (-.65,-3.35)
          (4-3.35,1.65) rectangle (4-.65,2.35)
          (4-3.35,2.65) rectangle (4-.65,3.35)
          (4-3.35,-1.65) rectangle (4-.65,-2.35)
          (4-3.35,-2.65) rectangle (4-.65,-3.35);


          % element labels (the $b^i_j$'s)
          \draw 
          %%% B1
          (-3,3) node {$b^1_8$}   (-2,3) node {$b^1_7$}   (-1,3) node {$b^1_6$}
          (-3,2) node {$b^1_5$}   (-2,2) node {$b^1_4$}   (-1,2) node {$b^1_3$}
          (-3,1) node {$b^1_2$}   (-2,1) node {$b^1_1$}   (-1,1) node {$b_0$}

          ( 0,1) node {$b_1$}
          
          %%% B2
          (1,3) node {$b^2_8$}  (2,3) node {$b^2_7$}   (3,3) node {$b^2_6$}
          (1,2) node {$b^2_5$}  (2,2) node {$b^2_4$}   (3,2) node {$b^2_3$}
          (1,1) node {$b_2$}    (2,1) node {$b^2_1$}   (3,1) node {$b^2_0$}

          %%% center block
          (-1,0) node {$b_3$}   ( 0,0) node {$b_4$}    ( 1,0) node {$b_5$}
          
          %%% B3
          (-3,-1) node {$b^3_8$}  (-2,-1) node {$b^3_7$}  (-1,-1) node {$b_6$}
          (-3,-2) node {$b^3_5$}  (-2,-2) node {$b^3_4$}  (-1,-2) node {$b^3_3$}
          (-3,-3) node {$b^3_2$}  (-2,-3) node {$b^3_1$}  (-1,-3) node {$b^3_0$}

          (0,-1) node {$b_7$}

          %%% B4
          (1,-1) node {$b_8$}   (2,-1) node {$b^4_7$}   (3,-1) node {$b^4_6$}
          (1,-3) node {$b^4_2$} (2,-3) node {$b^4_1$}   (3,-3) node {$b^4_0$}
          (1,-2) node {$b^4_5$} (2,-2) node {$b^4_4$}   (3,-2) node {$b^4_3$};


          \draw (0, -4) node {$\beta^\star$ $(=\beta^*)$};
        \end{tikzpicture}
        \hskip1cm
        \begin{tikzpicture}[scale=.7]
          % lines around the minimal universes (the $B_i$'s)
          \draw[sets]  
          (-1.5,-1.5) rectangle (1.5,1.5) % Center
          (.5,.5) rectangle (3.5,3.5)     % NE
          (.5,-3.5) rectangle (3.5,-.5)   % SE
          (-3.5,-3.5) rectangle (-.5,-.5) % SW
          (-3.5,.5) rectangle (-.5,3.5);  % NW


          % lines around equivalence classes 
          \draw[blocks] 
          (-1.35,-.35) rectangle (1.35,.35) % Center
          (-3.35,.65) rectangle (3.35,1.35)   % b1(0:2) b2(0:2)
          (-3.35,-.65) rectangle (3.35,-1.35) % b3(6:8) b4(6:8)
          (-3.35,1.65) rectangle (4-.65,2.35)
          (-3.35,2.65) rectangle (4-.65,3.35)
          (-3.35,-1.65) rectangle (4-.65,-2.35)
          (-3.35,-2.65) rectangle (4-.65,-3.35);


          % element labels (the $b^i_j$'s)
          \draw 
          %%% B1
          (-3,3) node {$b^1_8$}   (-2,3) node {$b^1_7$}   (-1,3) node {$b^1_6$}
          (-3,2) node {$b^1_5$}   (-2,2) node {$b^1_4$}   (-1,2) node {$b^1_3$}
          (-3,1) node {$b^1_2$}   (-2,1) node {$b^1_1$}   (-1,1) node {$b_0$}

          ( 0,1) node {$b_1$}
          
          %%% B2
          (1,3) node {$b^2_8$}  (2,3) node {$b^2_7$}   (3,3) node {$b^2_6$}
          (1,2) node {$b^2_5$}  (2,2) node {$b^2_4$}   (3,2) node {$b^2_3$}
          (1,1) node {$b_2$}    (2,1) node {$b^2_1$}   (3,1) node {$b^2_0$}

          %%% center block
          (-1,0) node {$b_3$}   ( 0,0) node {$b_4$}    ( 1,0) node {$b_5$}
          
          %%% B3
          (-3,-1) node {$b^3_8$}  (-2,-1) node {$b^3_7$}  (-1,-1) node {$b_6$}
          (-3,-2) node {$b^3_5$}  (-2,-2) node {$b^3_4$}  (-1,-2) node {$b^3_3$}
          (-3,-3) node {$b^3_2$}  (-2,-3) node {$b^3_1$}  (-1,-3) node {$b^3_0$}

          (0,-1) node {$b_7$}

          %%% B4
          (1,-1) node {$b_8$}   (2,-1) node {$b^4_7$}   (3,-1) node {$b^4_6$}
          (1,-3) node {$b^4_2$} (2,-3) node {$b^4_1$}   (3,-3) node {$b^4_0$}
          (1,-2) node {$b^4_5$} (2,-2) node {$b^4_4$}   (3,-2) node {$b^4_3$};


          \draw (0, -4) node {$\tbeta$ $(=\hbeta)$};
        \end{tikzpicture}
      }
      \caption{\mycaption}
      %% \label{fig:overalgebra1}
\end{figure}

\noindent 
The second option was used in the original paper. See \\
\url{https://github.com/williamdemeo/Overalgebras}.
\end{document}

