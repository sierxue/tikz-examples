%%% FILE: tripartite_graphs.tex
%%% DESCRIPTION: tikz code for some tripartite graphs
%%% AUTHOR: williamdemeo@gmail.com
%%% DATE: 2016-03-06
%%%
%%% DEPENDENCIES: `newpotatoes.tex` (input file defining the placement of the nodes)
%%%               `\smallscale` (latex macro specifying how big the diagrams should be)
%%%               `lat` (tikzstyle specifying default node style)
%%%               `myarrow` (tikzstyle specifying default arrow style)
%%%
%%% NOTES: This file (tripartite_graphs.tex) merely specifies lines between nodes.
%%%        The real work is done in the input file `newpotatoes.tex`, which specifies 
%%%        the layout of the vertices of the tripartites graph(s).  
%%%
%%%        Of course it's simpler to keep all the code to generate a given
%%%        figure in a single flat file, but the modular set-up we use here is better
%%%        for generating a bunch of graphs that use the same collections (``potatoes'')
%%%        of partite sets, while varying the edge set.
%%%
\documentclass{article}
\usepackage[top=3cm,bottom=3cm,left=3.00cm,right=3.00cm]{geometry}

\usepackage{tikz,url}
\usepackage{tkz-graph}
\usetikzlibrary{positioning,chains,fit,shapes,calc}

\usepackage{xcolor}
\usepackage[colorlinks=true,urlcolor=blue,linkcolor=blue,citecolor=blue]{hyperref}

\begin{document}

\tikzstyle{lat} = [circle,draw,inner sep=1pt]
\tikzstyle{myarrow} = [->,>=stealth',shorten >=1pt,shorten >=5pt,shorten <=7pt]
\definecolor{myblue}{RGB}{80,80,160}
\definecolor{mygreen}{RGB}{80,160,80}

\newcommand{\smallscale}{.8}

    \begin{tikzpicture}[scale=\smallscale]

      %% Vertices:
      %%% FILE: newpotatoes.tex
%%% DESCRIPTION: input file (tikz code for tripartite graphs)
%%% AUTHOR: williamdemeo@gmail.com
%%% DATE: 2016-03-06
%%%
%%% To insert the code in this file in your own tikzpicture environment, make
%%% sure this file is in your working directory or path and put some lines
%%% like the following in your LaTeX document:
%%%
%%%     \begin{tikzpicture}[scale=0.8]  %% <-- (change the scale if you want)
%%%        %%% FILE: newpotatoes.tex
%%% DESCRIPTION: input file (tikz code for tripartite graphs)
%%% AUTHOR: williamdemeo@gmail.com
%%% DATE: 2016-03-06
%%%
%%% To insert the code in this file in your own tikzpicture environment, make
%%% sure this file is in your working directory or path and put some lines
%%% like the following in your LaTeX document:
%%%
%%%     \begin{tikzpicture}[scale=0.8]  %% <-- (change the scale if you want)
%%%        %%% FILE: newpotatoes.tex
%%% DESCRIPTION: input file (tikz code for tripartite graphs)
%%% AUTHOR: williamdemeo@gmail.com
%%% DATE: 2016-03-06
%%%
%%% To insert the code in this file in your own tikzpicture environment, make
%%% sure this file is in your working directory or path and put some lines
%%% like the following in your LaTeX document:
%%%
%%%     \begin{tikzpicture}[scale=0.8]  %% <-- (change the scale if you want)
%%%        \input{newpotatoes}
%%%     \end{tikzpicture}
%%%
%%% See the file tripartite_graphs.tex for more examples.

%% Specify vertex style here:
\tikzstyle{vstyle}=[circle,draw=blue!50,fill=blue!20,thick,inner sep=0pt,minimum size=3mm]

%% Vertices in North-Western potato:
\foreach \i in {0,1,2,3}
\node[vstyle] (a\i) at (\i,1.25*\i) {\i};

%% Vertices in North-Eastern potato:
\foreach \i in {0,1,2,3}
\node[vstyle] (b\i) at (8.25-\i,1.25*\i) {\i};

%% Vertices in Southern potato:
\foreach \i in {0,1,2,3}
\node[vstyle] (c\i) at (2*\i+1,-2) {\i};

%% ellipse surrounding vertices in NW potato
\draw[dotted,thick] let \p1=(a0), \p2=(a3), \n1={atan2(\y2-\y1,\x2-\x1)}, \n2={veclen(\y2-\y1,\x2-\x1)}
in ($ (a0)!0.5!(a3) $) ellipse [x radius=\n2/2+20pt, y radius=0.7cm,rotate=90-\n1];

%% ellipse surrounding vertices in NE potato
\draw[dotted,thick] let \p1=(b0), \p2=(b3), \n1={atan2(\y2-\y1,\x2-\x1)}, \n2={veclen(\y2-\y1,\x2-\x1)}
in ($ (b0)!0.5!(b3) $) ellipse [x radius=\n2/2+20pt, y radius=0.7cm,rotate=90-\n1];

%% ellipse surrounding vertices in Southern potato
\draw[dotted,thick] let \p1=(c0), \p2=(c3), \n1={atan2(\y2-\y1,\x2-\x1)}, \n2={veclen(\y2-\y1,\x2-\x1)}
in ($ (c0)!0.5!(c3) $) ellipse [x radius=\n2/2+20pt, y radius=0.7cm]; %,rotate=90-\n1];

%%%     \end{tikzpicture}
%%%
%%% See the file tripartite_graphs.tex for more examples.

%% Specify vertex style here:
\tikzstyle{vstyle}=[circle,draw=blue!50,fill=blue!20,thick,inner sep=0pt,minimum size=3mm]

%% Vertices in North-Western potato:
\foreach \i in {0,1,2,3}
\node[vstyle] (a\i) at (\i,1.25*\i) {\i};

%% Vertices in North-Eastern potato:
\foreach \i in {0,1,2,3}
\node[vstyle] (b\i) at (8.25-\i,1.25*\i) {\i};

%% Vertices in Southern potato:
\foreach \i in {0,1,2,3}
\node[vstyle] (c\i) at (2*\i+1,-2) {\i};

%% ellipse surrounding vertices in NW potato
\draw[dotted,thick] let \p1=(a0), \p2=(a3), \n1={atan2(\y2-\y1,\x2-\x1)}, \n2={veclen(\y2-\y1,\x2-\x1)}
in ($ (a0)!0.5!(a3) $) ellipse [x radius=\n2/2+20pt, y radius=0.7cm,rotate=90-\n1];

%% ellipse surrounding vertices in NE potato
\draw[dotted,thick] let \p1=(b0), \p2=(b3), \n1={atan2(\y2-\y1,\x2-\x1)}, \n2={veclen(\y2-\y1,\x2-\x1)}
in ($ (b0)!0.5!(b3) $) ellipse [x radius=\n2/2+20pt, y radius=0.7cm,rotate=90-\n1];

%% ellipse surrounding vertices in Southern potato
\draw[dotted,thick] let \p1=(c0), \p2=(c3), \n1={atan2(\y2-\y1,\x2-\x1)}, \n2={veclen(\y2-\y1,\x2-\x1)}
in ($ (c0)!0.5!(c3) $) ellipse [x radius=\n2/2+20pt, y radius=0.7cm]; %,rotate=90-\n1];

%%%     \end{tikzpicture}
%%%
%%% See the file tripartite_graphs.tex for more examples.

%% Specify vertex style here:
\tikzstyle{vstyle}=[circle,draw=blue!50,fill=blue!20,thick,inner sep=0pt,minimum size=3mm]

%% Vertices in North-Western potato:
\foreach \i in {0,1,2,3}
\node[vstyle] (a\i) at (\i,1.25*\i) {\i};

%% Vertices in North-Eastern potato:
\foreach \i in {0,1,2,3}
\node[vstyle] (b\i) at (8.25-\i,1.25*\i) {\i};

%% Vertices in Southern potato:
\foreach \i in {0,1,2,3}
\node[vstyle] (c\i) at (2*\i+1,-2) {\i};

%% ellipse surrounding vertices in NW potato
\draw[dotted,thick] let \p1=(a0), \p2=(a3), \n1={atan2(\y2-\y1,\x2-\x1)}, \n2={veclen(\y2-\y1,\x2-\x1)}
in ($ (a0)!0.5!(a3) $) ellipse [x radius=\n2/2+20pt, y radius=0.7cm,rotate=90-\n1];

%% ellipse surrounding vertices in NE potato
\draw[dotted,thick] let \p1=(b0), \p2=(b3), \n1={atan2(\y2-\y1,\x2-\x1)}, \n2={veclen(\y2-\y1,\x2-\x1)}
in ($ (b0)!0.5!(b3) $) ellipse [x radius=\n2/2+20pt, y radius=0.7cm,rotate=90-\n1];

%% ellipse surrounding vertices in Southern potato
\draw[dotted,thick] let \p1=(c0), \p2=(c3), \n1={atan2(\y2-\y1,\x2-\x1)}, \n2={veclen(\y2-\y1,\x2-\x1)}
in ($ (c0)!0.5!(c3) $) ellipse [x radius=\n2/2+20pt, y radius=0.7cm]; %,rotate=90-\n1];


      %% Edges: {0, 1, 2, 3}
      \begin{scope}[very thick,every node/.style={midway,fill=white}]
        \draw[red]     (a1)--(b0)--(c0)--(a1) node {2};
        \draw[black]   (a0)--(b0) node {0} (b0)--(c1)--(a0) ;
        \draw[myblue]    (a0)--(b1)--(c0)--(a0) node {1};
        \draw[mygreen] (a1)--(b1) node {3} (b1)--(c1)--(a1) ;
      \end{scope}

    \end{tikzpicture}
    %% ------------------------------------------------------
    \hskip5mm 
    %% ------------------------------------------------------
    \begin{tikzpicture}[scale=\smallscale]

      %% Vertices:
      %%% FILE: newpotatoes.tex
%%% DESCRIPTION: input file (tikz code for tripartite graphs)
%%% AUTHOR: williamdemeo@gmail.com
%%% DATE: 2016-03-06
%%%
%%% To insert the code in this file in your own tikzpicture environment, make
%%% sure this file is in your working directory or path and put some lines
%%% like the following in your LaTeX document:
%%%
%%%     \begin{tikzpicture}[scale=0.8]  %% <-- (change the scale if you want)
%%%        %%% FILE: newpotatoes.tex
%%% DESCRIPTION: input file (tikz code for tripartite graphs)
%%% AUTHOR: williamdemeo@gmail.com
%%% DATE: 2016-03-06
%%%
%%% To insert the code in this file in your own tikzpicture environment, make
%%% sure this file is in your working directory or path and put some lines
%%% like the following in your LaTeX document:
%%%
%%%     \begin{tikzpicture}[scale=0.8]  %% <-- (change the scale if you want)
%%%        %%% FILE: newpotatoes.tex
%%% DESCRIPTION: input file (tikz code for tripartite graphs)
%%% AUTHOR: williamdemeo@gmail.com
%%% DATE: 2016-03-06
%%%
%%% To insert the code in this file in your own tikzpicture environment, make
%%% sure this file is in your working directory or path and put some lines
%%% like the following in your LaTeX document:
%%%
%%%     \begin{tikzpicture}[scale=0.8]  %% <-- (change the scale if you want)
%%%        \input{newpotatoes}
%%%     \end{tikzpicture}
%%%
%%% See the file tripartite_graphs.tex for more examples.

%% Specify vertex style here:
\tikzstyle{vstyle}=[circle,draw=blue!50,fill=blue!20,thick,inner sep=0pt,minimum size=3mm]

%% Vertices in North-Western potato:
\foreach \i in {0,1,2,3}
\node[vstyle] (a\i) at (\i,1.25*\i) {\i};

%% Vertices in North-Eastern potato:
\foreach \i in {0,1,2,3}
\node[vstyle] (b\i) at (8.25-\i,1.25*\i) {\i};

%% Vertices in Southern potato:
\foreach \i in {0,1,2,3}
\node[vstyle] (c\i) at (2*\i+1,-2) {\i};

%% ellipse surrounding vertices in NW potato
\draw[dotted,thick] let \p1=(a0), \p2=(a3), \n1={atan2(\y2-\y1,\x2-\x1)}, \n2={veclen(\y2-\y1,\x2-\x1)}
in ($ (a0)!0.5!(a3) $) ellipse [x radius=\n2/2+20pt, y radius=0.7cm,rotate=90-\n1];

%% ellipse surrounding vertices in NE potato
\draw[dotted,thick] let \p1=(b0), \p2=(b3), \n1={atan2(\y2-\y1,\x2-\x1)}, \n2={veclen(\y2-\y1,\x2-\x1)}
in ($ (b0)!0.5!(b3) $) ellipse [x radius=\n2/2+20pt, y radius=0.7cm,rotate=90-\n1];

%% ellipse surrounding vertices in Southern potato
\draw[dotted,thick] let \p1=(c0), \p2=(c3), \n1={atan2(\y2-\y1,\x2-\x1)}, \n2={veclen(\y2-\y1,\x2-\x1)}
in ($ (c0)!0.5!(c3) $) ellipse [x radius=\n2/2+20pt, y radius=0.7cm]; %,rotate=90-\n1];

%%%     \end{tikzpicture}
%%%
%%% See the file tripartite_graphs.tex for more examples.

%% Specify vertex style here:
\tikzstyle{vstyle}=[circle,draw=blue!50,fill=blue!20,thick,inner sep=0pt,minimum size=3mm]

%% Vertices in North-Western potato:
\foreach \i in {0,1,2,3}
\node[vstyle] (a\i) at (\i,1.25*\i) {\i};

%% Vertices in North-Eastern potato:
\foreach \i in {0,1,2,3}
\node[vstyle] (b\i) at (8.25-\i,1.25*\i) {\i};

%% Vertices in Southern potato:
\foreach \i in {0,1,2,3}
\node[vstyle] (c\i) at (2*\i+1,-2) {\i};

%% ellipse surrounding vertices in NW potato
\draw[dotted,thick] let \p1=(a0), \p2=(a3), \n1={atan2(\y2-\y1,\x2-\x1)}, \n2={veclen(\y2-\y1,\x2-\x1)}
in ($ (a0)!0.5!(a3) $) ellipse [x radius=\n2/2+20pt, y radius=0.7cm,rotate=90-\n1];

%% ellipse surrounding vertices in NE potato
\draw[dotted,thick] let \p1=(b0), \p2=(b3), \n1={atan2(\y2-\y1,\x2-\x1)}, \n2={veclen(\y2-\y1,\x2-\x1)}
in ($ (b0)!0.5!(b3) $) ellipse [x radius=\n2/2+20pt, y radius=0.7cm,rotate=90-\n1];

%% ellipse surrounding vertices in Southern potato
\draw[dotted,thick] let \p1=(c0), \p2=(c3), \n1={atan2(\y2-\y1,\x2-\x1)}, \n2={veclen(\y2-\y1,\x2-\x1)}
in ($ (c0)!0.5!(c3) $) ellipse [x radius=\n2/2+20pt, y radius=0.7cm]; %,rotate=90-\n1];

%%%     \end{tikzpicture}
%%%
%%% See the file tripartite_graphs.tex for more examples.

%% Specify vertex style here:
\tikzstyle{vstyle}=[circle,draw=blue!50,fill=blue!20,thick,inner sep=0pt,minimum size=3mm]

%% Vertices in North-Western potato:
\foreach \i in {0,1,2,3}
\node[vstyle] (a\i) at (\i,1.25*\i) {\i};

%% Vertices in North-Eastern potato:
\foreach \i in {0,1,2,3}
\node[vstyle] (b\i) at (8.25-\i,1.25*\i) {\i};

%% Vertices in Southern potato:
\foreach \i in {0,1,2,3}
\node[vstyle] (c\i) at (2*\i+1,-2) {\i};

%% ellipse surrounding vertices in NW potato
\draw[dotted,thick] let \p1=(a0), \p2=(a3), \n1={atan2(\y2-\y1,\x2-\x1)}, \n2={veclen(\y2-\y1,\x2-\x1)}
in ($ (a0)!0.5!(a3) $) ellipse [x radius=\n2/2+20pt, y radius=0.7cm,rotate=90-\n1];

%% ellipse surrounding vertices in NE potato
\draw[dotted,thick] let \p1=(b0), \p2=(b3), \n1={atan2(\y2-\y1,\x2-\x1)}, \n2={veclen(\y2-\y1,\x2-\x1)}
in ($ (b0)!0.5!(b3) $) ellipse [x radius=\n2/2+20pt, y radius=0.7cm,rotate=90-\n1];

%% ellipse surrounding vertices in Southern potato
\draw[dotted,thick] let \p1=(c0), \p2=(c3), \n1={atan2(\y2-\y1,\x2-\x1)}, \n2={veclen(\y2-\y1,\x2-\x1)}
in ($ (c0)!0.5!(c3) $) ellipse [x radius=\n2/2+20pt, y radius=0.7cm]; %,rotate=90-\n1];


      %% Edges: {4, 5, 6, 7}
      \begin{scope}[very thick,every node/.style={midway,fill=white}]
        \draw[blue]    (a0)--(b3)--(c3)--(a0) node[pos=.7] {5};
        \draw[black]   (a0)--(b2)--(c2)--(a0) node[pos=.4] {4};
        \draw[red]     (a1)--(b2)--(c3)--(a1) node {6};
        \draw[mygreen] (a1)--(b3)--(c2)--(a1) node {7};
      \end{scope}
    \end{tikzpicture}

    %% ======================================================
    \vskip2cm
    %% ======================================================

    \begin{tikzpicture}[scale=\smallscale]

      %% Vertices:
      %%% FILE: newpotatoes.tex
%%% DESCRIPTION: input file (tikz code for tripartite graphs)
%%% AUTHOR: williamdemeo@gmail.com
%%% DATE: 2016-03-06
%%%
%%% To insert the code in this file in your own tikzpicture environment, make
%%% sure this file is in your working directory or path and put some lines
%%% like the following in your LaTeX document:
%%%
%%%     \begin{tikzpicture}[scale=0.8]  %% <-- (change the scale if you want)
%%%        %%% FILE: newpotatoes.tex
%%% DESCRIPTION: input file (tikz code for tripartite graphs)
%%% AUTHOR: williamdemeo@gmail.com
%%% DATE: 2016-03-06
%%%
%%% To insert the code in this file in your own tikzpicture environment, make
%%% sure this file is in your working directory or path and put some lines
%%% like the following in your LaTeX document:
%%%
%%%     \begin{tikzpicture}[scale=0.8]  %% <-- (change the scale if you want)
%%%        %%% FILE: newpotatoes.tex
%%% DESCRIPTION: input file (tikz code for tripartite graphs)
%%% AUTHOR: williamdemeo@gmail.com
%%% DATE: 2016-03-06
%%%
%%% To insert the code in this file in your own tikzpicture environment, make
%%% sure this file is in your working directory or path and put some lines
%%% like the following in your LaTeX document:
%%%
%%%     \begin{tikzpicture}[scale=0.8]  %% <-- (change the scale if you want)
%%%        \input{newpotatoes}
%%%     \end{tikzpicture}
%%%
%%% See the file tripartite_graphs.tex for more examples.

%% Specify vertex style here:
\tikzstyle{vstyle}=[circle,draw=blue!50,fill=blue!20,thick,inner sep=0pt,minimum size=3mm]

%% Vertices in North-Western potato:
\foreach \i in {0,1,2,3}
\node[vstyle] (a\i) at (\i,1.25*\i) {\i};

%% Vertices in North-Eastern potato:
\foreach \i in {0,1,2,3}
\node[vstyle] (b\i) at (8.25-\i,1.25*\i) {\i};

%% Vertices in Southern potato:
\foreach \i in {0,1,2,3}
\node[vstyle] (c\i) at (2*\i+1,-2) {\i};

%% ellipse surrounding vertices in NW potato
\draw[dotted,thick] let \p1=(a0), \p2=(a3), \n1={atan2(\y2-\y1,\x2-\x1)}, \n2={veclen(\y2-\y1,\x2-\x1)}
in ($ (a0)!0.5!(a3) $) ellipse [x radius=\n2/2+20pt, y radius=0.7cm,rotate=90-\n1];

%% ellipse surrounding vertices in NE potato
\draw[dotted,thick] let \p1=(b0), \p2=(b3), \n1={atan2(\y2-\y1,\x2-\x1)}, \n2={veclen(\y2-\y1,\x2-\x1)}
in ($ (b0)!0.5!(b3) $) ellipse [x radius=\n2/2+20pt, y radius=0.7cm,rotate=90-\n1];

%% ellipse surrounding vertices in Southern potato
\draw[dotted,thick] let \p1=(c0), \p2=(c3), \n1={atan2(\y2-\y1,\x2-\x1)}, \n2={veclen(\y2-\y1,\x2-\x1)}
in ($ (c0)!0.5!(c3) $) ellipse [x radius=\n2/2+20pt, y radius=0.7cm]; %,rotate=90-\n1];

%%%     \end{tikzpicture}
%%%
%%% See the file tripartite_graphs.tex for more examples.

%% Specify vertex style here:
\tikzstyle{vstyle}=[circle,draw=blue!50,fill=blue!20,thick,inner sep=0pt,minimum size=3mm]

%% Vertices in North-Western potato:
\foreach \i in {0,1,2,3}
\node[vstyle] (a\i) at (\i,1.25*\i) {\i};

%% Vertices in North-Eastern potato:
\foreach \i in {0,1,2,3}
\node[vstyle] (b\i) at (8.25-\i,1.25*\i) {\i};

%% Vertices in Southern potato:
\foreach \i in {0,1,2,3}
\node[vstyle] (c\i) at (2*\i+1,-2) {\i};

%% ellipse surrounding vertices in NW potato
\draw[dotted,thick] let \p1=(a0), \p2=(a3), \n1={atan2(\y2-\y1,\x2-\x1)}, \n2={veclen(\y2-\y1,\x2-\x1)}
in ($ (a0)!0.5!(a3) $) ellipse [x radius=\n2/2+20pt, y radius=0.7cm,rotate=90-\n1];

%% ellipse surrounding vertices in NE potato
\draw[dotted,thick] let \p1=(b0), \p2=(b3), \n1={atan2(\y2-\y1,\x2-\x1)}, \n2={veclen(\y2-\y1,\x2-\x1)}
in ($ (b0)!0.5!(b3) $) ellipse [x radius=\n2/2+20pt, y radius=0.7cm,rotate=90-\n1];

%% ellipse surrounding vertices in Southern potato
\draw[dotted,thick] let \p1=(c0), \p2=(c3), \n1={atan2(\y2-\y1,\x2-\x1)}, \n2={veclen(\y2-\y1,\x2-\x1)}
in ($ (c0)!0.5!(c3) $) ellipse [x radius=\n2/2+20pt, y radius=0.7cm]; %,rotate=90-\n1];

%%%     \end{tikzpicture}
%%%
%%% See the file tripartite_graphs.tex for more examples.

%% Specify vertex style here:
\tikzstyle{vstyle}=[circle,draw=blue!50,fill=blue!20,thick,inner sep=0pt,minimum size=3mm]

%% Vertices in North-Western potato:
\foreach \i in {0,1,2,3}
\node[vstyle] (a\i) at (\i,1.25*\i) {\i};

%% Vertices in North-Eastern potato:
\foreach \i in {0,1,2,3}
\node[vstyle] (b\i) at (8.25-\i,1.25*\i) {\i};

%% Vertices in Southern potato:
\foreach \i in {0,1,2,3}
\node[vstyle] (c\i) at (2*\i+1,-2) {\i};

%% ellipse surrounding vertices in NW potato
\draw[dotted,thick] let \p1=(a0), \p2=(a3), \n1={atan2(\y2-\y1,\x2-\x1)}, \n2={veclen(\y2-\y1,\x2-\x1)}
in ($ (a0)!0.5!(a3) $) ellipse [x radius=\n2/2+20pt, y radius=0.7cm,rotate=90-\n1];

%% ellipse surrounding vertices in NE potato
\draw[dotted,thick] let \p1=(b0), \p2=(b3), \n1={atan2(\y2-\y1,\x2-\x1)}, \n2={veclen(\y2-\y1,\x2-\x1)}
in ($ (b0)!0.5!(b3) $) ellipse [x radius=\n2/2+20pt, y radius=0.7cm,rotate=90-\n1];

%% ellipse surrounding vertices in Southern potato
\draw[dotted,thick] let \p1=(c0), \p2=(c3), \n1={atan2(\y2-\y1,\x2-\x1)}, \n2={veclen(\y2-\y1,\x2-\x1)}
in ($ (c0)!0.5!(c3) $) ellipse [x radius=\n2/2+20pt, y radius=0.7cm]; %,rotate=90-\n1];


      %% Edges: {8, 9, 10, 11}
      \begin{scope}[very thick,every node/.style={midway,fill=white}]
        \draw[black]   (a2)--(b0)--(c2)--(a2) node {8};
        \draw[blue]    (a3)--(b0)--(c3)--(a3) node {9};
        \draw[red]     (a2)--(b1)--(c3)--(a2) node {10};
        \draw[mygreen] (a3)--(b1)--(c2) node[pos=.4] {11} (c2)--(a3);
      \end{scope}

    \end{tikzpicture}
    %% ------------------------------------------------------
    \hskip5mm 
    %% ------------------------------------------------------
    \begin{tikzpicture}[scale=\smallscale]

      %% Vertices:
      %%% FILE: newpotatoes.tex
%%% DESCRIPTION: input file (tikz code for tripartite graphs)
%%% AUTHOR: williamdemeo@gmail.com
%%% DATE: 2016-03-06
%%%
%%% To insert the code in this file in your own tikzpicture environment, make
%%% sure this file is in your working directory or path and put some lines
%%% like the following in your LaTeX document:
%%%
%%%     \begin{tikzpicture}[scale=0.8]  %% <-- (change the scale if you want)
%%%        %%% FILE: newpotatoes.tex
%%% DESCRIPTION: input file (tikz code for tripartite graphs)
%%% AUTHOR: williamdemeo@gmail.com
%%% DATE: 2016-03-06
%%%
%%% To insert the code in this file in your own tikzpicture environment, make
%%% sure this file is in your working directory or path and put some lines
%%% like the following in your LaTeX document:
%%%
%%%     \begin{tikzpicture}[scale=0.8]  %% <-- (change the scale if you want)
%%%        %%% FILE: newpotatoes.tex
%%% DESCRIPTION: input file (tikz code for tripartite graphs)
%%% AUTHOR: williamdemeo@gmail.com
%%% DATE: 2016-03-06
%%%
%%% To insert the code in this file in your own tikzpicture environment, make
%%% sure this file is in your working directory or path and put some lines
%%% like the following in your LaTeX document:
%%%
%%%     \begin{tikzpicture}[scale=0.8]  %% <-- (change the scale if you want)
%%%        \input{newpotatoes}
%%%     \end{tikzpicture}
%%%
%%% See the file tripartite_graphs.tex for more examples.

%% Specify vertex style here:
\tikzstyle{vstyle}=[circle,draw=blue!50,fill=blue!20,thick,inner sep=0pt,minimum size=3mm]

%% Vertices in North-Western potato:
\foreach \i in {0,1,2,3}
\node[vstyle] (a\i) at (\i,1.25*\i) {\i};

%% Vertices in North-Eastern potato:
\foreach \i in {0,1,2,3}
\node[vstyle] (b\i) at (8.25-\i,1.25*\i) {\i};

%% Vertices in Southern potato:
\foreach \i in {0,1,2,3}
\node[vstyle] (c\i) at (2*\i+1,-2) {\i};

%% ellipse surrounding vertices in NW potato
\draw[dotted,thick] let \p1=(a0), \p2=(a3), \n1={atan2(\y2-\y1,\x2-\x1)}, \n2={veclen(\y2-\y1,\x2-\x1)}
in ($ (a0)!0.5!(a3) $) ellipse [x radius=\n2/2+20pt, y radius=0.7cm,rotate=90-\n1];

%% ellipse surrounding vertices in NE potato
\draw[dotted,thick] let \p1=(b0), \p2=(b3), \n1={atan2(\y2-\y1,\x2-\x1)}, \n2={veclen(\y2-\y1,\x2-\x1)}
in ($ (b0)!0.5!(b3) $) ellipse [x radius=\n2/2+20pt, y radius=0.7cm,rotate=90-\n1];

%% ellipse surrounding vertices in Southern potato
\draw[dotted,thick] let \p1=(c0), \p2=(c3), \n1={atan2(\y2-\y1,\x2-\x1)}, \n2={veclen(\y2-\y1,\x2-\x1)}
in ($ (c0)!0.5!(c3) $) ellipse [x radius=\n2/2+20pt, y radius=0.7cm]; %,rotate=90-\n1];

%%%     \end{tikzpicture}
%%%
%%% See the file tripartite_graphs.tex for more examples.

%% Specify vertex style here:
\tikzstyle{vstyle}=[circle,draw=blue!50,fill=blue!20,thick,inner sep=0pt,minimum size=3mm]

%% Vertices in North-Western potato:
\foreach \i in {0,1,2,3}
\node[vstyle] (a\i) at (\i,1.25*\i) {\i};

%% Vertices in North-Eastern potato:
\foreach \i in {0,1,2,3}
\node[vstyle] (b\i) at (8.25-\i,1.25*\i) {\i};

%% Vertices in Southern potato:
\foreach \i in {0,1,2,3}
\node[vstyle] (c\i) at (2*\i+1,-2) {\i};

%% ellipse surrounding vertices in NW potato
\draw[dotted,thick] let \p1=(a0), \p2=(a3), \n1={atan2(\y2-\y1,\x2-\x1)}, \n2={veclen(\y2-\y1,\x2-\x1)}
in ($ (a0)!0.5!(a3) $) ellipse [x radius=\n2/2+20pt, y radius=0.7cm,rotate=90-\n1];

%% ellipse surrounding vertices in NE potato
\draw[dotted,thick] let \p1=(b0), \p2=(b3), \n1={atan2(\y2-\y1,\x2-\x1)}, \n2={veclen(\y2-\y1,\x2-\x1)}
in ($ (b0)!0.5!(b3) $) ellipse [x radius=\n2/2+20pt, y radius=0.7cm,rotate=90-\n1];

%% ellipse surrounding vertices in Southern potato
\draw[dotted,thick] let \p1=(c0), \p2=(c3), \n1={atan2(\y2-\y1,\x2-\x1)}, \n2={veclen(\y2-\y1,\x2-\x1)}
in ($ (c0)!0.5!(c3) $) ellipse [x radius=\n2/2+20pt, y radius=0.7cm]; %,rotate=90-\n1];

%%%     \end{tikzpicture}
%%%
%%% See the file tripartite_graphs.tex for more examples.

%% Specify vertex style here:
\tikzstyle{vstyle}=[circle,draw=blue!50,fill=blue!20,thick,inner sep=0pt,minimum size=3mm]

%% Vertices in North-Western potato:
\foreach \i in {0,1,2,3}
\node[vstyle] (a\i) at (\i,1.25*\i) {\i};

%% Vertices in North-Eastern potato:
\foreach \i in {0,1,2,3}
\node[vstyle] (b\i) at (8.25-\i,1.25*\i) {\i};

%% Vertices in Southern potato:
\foreach \i in {0,1,2,3}
\node[vstyle] (c\i) at (2*\i+1,-2) {\i};

%% ellipse surrounding vertices in NW potato
\draw[dotted,thick] let \p1=(a0), \p2=(a3), \n1={atan2(\y2-\y1,\x2-\x1)}, \n2={veclen(\y2-\y1,\x2-\x1)}
in ($ (a0)!0.5!(a3) $) ellipse [x radius=\n2/2+20pt, y radius=0.7cm,rotate=90-\n1];

%% ellipse surrounding vertices in NE potato
\draw[dotted,thick] let \p1=(b0), \p2=(b3), \n1={atan2(\y2-\y1,\x2-\x1)}, \n2={veclen(\y2-\y1,\x2-\x1)}
in ($ (b0)!0.5!(b3) $) ellipse [x radius=\n2/2+20pt, y radius=0.7cm,rotate=90-\n1];

%% ellipse surrounding vertices in Southern potato
\draw[dotted,thick] let \p1=(c0), \p2=(c3), \n1={atan2(\y2-\y1,\x2-\x1)}, \n2={veclen(\y2-\y1,\x2-\x1)}
in ($ (c0)!0.5!(c3) $) ellipse [x radius=\n2/2+20pt, y radius=0.7cm]; %,rotate=90-\n1];


      %% Edges: {8, 9, 10, 11}
      \begin{scope}[very thick,every node/.style={pos=.4,fill=white}]
        \draw[red]     (a2)--(b3)--(c1) node {14} (c1)--(a2);
        \draw[black]   (a2)--(b2) node {12} (b2)--(c0)--(a2);
        \draw[blue]    (a3)--(b3)--(c0) node {13} (c0)--(a3);
        \draw[mygreen] (a3)--(b2)--(c1) node {15} (c1)--(a3);
      \end{scope}

    \end{tikzpicture}

\newpage


    \begin{tikzpicture}[scale=1.5]

      %% Vertices:
      %%% FILE: newpotatoes.tex
%%% DESCRIPTION: input file (tikz code for tripartite graphs)
%%% AUTHOR: williamdemeo@gmail.com
%%% DATE: 2016-03-06
%%%
%%% To insert the code in this file in your own tikzpicture environment, make
%%% sure this file is in your working directory or path and put some lines
%%% like the following in your LaTeX document:
%%%
%%%     \begin{tikzpicture}[scale=0.8]  %% <-- (change the scale if you want)
%%%        %%% FILE: newpotatoes.tex
%%% DESCRIPTION: input file (tikz code for tripartite graphs)
%%% AUTHOR: williamdemeo@gmail.com
%%% DATE: 2016-03-06
%%%
%%% To insert the code in this file in your own tikzpicture environment, make
%%% sure this file is in your working directory or path and put some lines
%%% like the following in your LaTeX document:
%%%
%%%     \begin{tikzpicture}[scale=0.8]  %% <-- (change the scale if you want)
%%%        %%% FILE: newpotatoes.tex
%%% DESCRIPTION: input file (tikz code for tripartite graphs)
%%% AUTHOR: williamdemeo@gmail.com
%%% DATE: 2016-03-06
%%%
%%% To insert the code in this file in your own tikzpicture environment, make
%%% sure this file is in your working directory or path and put some lines
%%% like the following in your LaTeX document:
%%%
%%%     \begin{tikzpicture}[scale=0.8]  %% <-- (change the scale if you want)
%%%        \input{newpotatoes}
%%%     \end{tikzpicture}
%%%
%%% See the file tripartite_graphs.tex for more examples.

%% Specify vertex style here:
\tikzstyle{vstyle}=[circle,draw=blue!50,fill=blue!20,thick,inner sep=0pt,minimum size=3mm]

%% Vertices in North-Western potato:
\foreach \i in {0,1,2,3}
\node[vstyle] (a\i) at (\i,1.25*\i) {\i};

%% Vertices in North-Eastern potato:
\foreach \i in {0,1,2,3}
\node[vstyle] (b\i) at (8.25-\i,1.25*\i) {\i};

%% Vertices in Southern potato:
\foreach \i in {0,1,2,3}
\node[vstyle] (c\i) at (2*\i+1,-2) {\i};

%% ellipse surrounding vertices in NW potato
\draw[dotted,thick] let \p1=(a0), \p2=(a3), \n1={atan2(\y2-\y1,\x2-\x1)}, \n2={veclen(\y2-\y1,\x2-\x1)}
in ($ (a0)!0.5!(a3) $) ellipse [x radius=\n2/2+20pt, y radius=0.7cm,rotate=90-\n1];

%% ellipse surrounding vertices in NE potato
\draw[dotted,thick] let \p1=(b0), \p2=(b3), \n1={atan2(\y2-\y1,\x2-\x1)}, \n2={veclen(\y2-\y1,\x2-\x1)}
in ($ (b0)!0.5!(b3) $) ellipse [x radius=\n2/2+20pt, y radius=0.7cm,rotate=90-\n1];

%% ellipse surrounding vertices in Southern potato
\draw[dotted,thick] let \p1=(c0), \p2=(c3), \n1={atan2(\y2-\y1,\x2-\x1)}, \n2={veclen(\y2-\y1,\x2-\x1)}
in ($ (c0)!0.5!(c3) $) ellipse [x radius=\n2/2+20pt, y radius=0.7cm]; %,rotate=90-\n1];

%%%     \end{tikzpicture}
%%%
%%% See the file tripartite_graphs.tex for more examples.

%% Specify vertex style here:
\tikzstyle{vstyle}=[circle,draw=blue!50,fill=blue!20,thick,inner sep=0pt,minimum size=3mm]

%% Vertices in North-Western potato:
\foreach \i in {0,1,2,3}
\node[vstyle] (a\i) at (\i,1.25*\i) {\i};

%% Vertices in North-Eastern potato:
\foreach \i in {0,1,2,3}
\node[vstyle] (b\i) at (8.25-\i,1.25*\i) {\i};

%% Vertices in Southern potato:
\foreach \i in {0,1,2,3}
\node[vstyle] (c\i) at (2*\i+1,-2) {\i};

%% ellipse surrounding vertices in NW potato
\draw[dotted,thick] let \p1=(a0), \p2=(a3), \n1={atan2(\y2-\y1,\x2-\x1)}, \n2={veclen(\y2-\y1,\x2-\x1)}
in ($ (a0)!0.5!(a3) $) ellipse [x radius=\n2/2+20pt, y radius=0.7cm,rotate=90-\n1];

%% ellipse surrounding vertices in NE potato
\draw[dotted,thick] let \p1=(b0), \p2=(b3), \n1={atan2(\y2-\y1,\x2-\x1)}, \n2={veclen(\y2-\y1,\x2-\x1)}
in ($ (b0)!0.5!(b3) $) ellipse [x radius=\n2/2+20pt, y radius=0.7cm,rotate=90-\n1];

%% ellipse surrounding vertices in Southern potato
\draw[dotted,thick] let \p1=(c0), \p2=(c3), \n1={atan2(\y2-\y1,\x2-\x1)}, \n2={veclen(\y2-\y1,\x2-\x1)}
in ($ (c0)!0.5!(c3) $) ellipse [x radius=\n2/2+20pt, y radius=0.7cm]; %,rotate=90-\n1];

%%%     \end{tikzpicture}
%%%
%%% See the file tripartite_graphs.tex for more examples.

%% Specify vertex style here:
\tikzstyle{vstyle}=[circle,draw=blue!50,fill=blue!20,thick,inner sep=0pt,minimum size=3mm]

%% Vertices in North-Western potato:
\foreach \i in {0,1,2,3}
\node[vstyle] (a\i) at (\i,1.25*\i) {\i};

%% Vertices in North-Eastern potato:
\foreach \i in {0,1,2,3}
\node[vstyle] (b\i) at (8.25-\i,1.25*\i) {\i};

%% Vertices in Southern potato:
\foreach \i in {0,1,2,3}
\node[vstyle] (c\i) at (2*\i+1,-2) {\i};

%% ellipse surrounding vertices in NW potato
\draw[dotted,thick] let \p1=(a0), \p2=(a3), \n1={atan2(\y2-\y1,\x2-\x1)}, \n2={veclen(\y2-\y1,\x2-\x1)}
in ($ (a0)!0.5!(a3) $) ellipse [x radius=\n2/2+20pt, y radius=0.7cm,rotate=90-\n1];

%% ellipse surrounding vertices in NE potato
\draw[dotted,thick] let \p1=(b0), \p2=(b3), \n1={atan2(\y2-\y1,\x2-\x1)}, \n2={veclen(\y2-\y1,\x2-\x1)}
in ($ (b0)!0.5!(b3) $) ellipse [x radius=\n2/2+20pt, y radius=0.7cm,rotate=90-\n1];

%% ellipse surrounding vertices in Southern potato
\draw[dotted,thick] let \p1=(c0), \p2=(c3), \n1={atan2(\y2-\y1,\x2-\x1)}, \n2={veclen(\y2-\y1,\x2-\x1)}
in ($ (c0)!0.5!(c3) $) ellipse [x radius=\n2/2+20pt, y radius=0.7cm]; %,rotate=90-\n1];


      %% Edges: (all edges with colors arranged in cycles)
      \begin{scope}[very thick]
        \draw[black]   (a0)--(b0)--(c1)--(a0);
        \draw[black]    (a0)--(b1)--(c0)--(a0);
        \draw[black]     (a1)--(b0)--(c0)--(a1);
        \draw[black] (a1)--(b1)--(c1)--(a1);
        \draw[blue]   (a0)--(b2)--(c2)--(a0);
        \draw[blue]    (a0)--(b3)--(c3)--(a0);
        \draw[blue]     (a1)--(b2)--(c3)--(a1);
        \draw[blue] (a1)--(b3)--(c2)--(a1);
        \draw[red]   (a2)--(b0)--(c2)--(a2);
        \draw[red]    (a3)--(b0)--(c3)--(a3);
        \draw[red]     (a2)--(b1)--(c3)--(a2);
        \draw[red]     (a3)--(b1)--(c2)--(a3);
        \draw[mygreen]   (a2)--(b2)--(c0)--(a2);
        \draw[mygreen]    (a3)--(b3)--(c0)--(a3);
        \draw[mygreen]     (a2)--(b3)--(c1)--(a2);
        \draw[mygreen]      (a3)--(b2)--(c1)--(a3);
      \end{scope}

    \end{tikzpicture}

\newpage

    \begin{tikzpicture}[scale=1.5]

      %% Vertices:
      %%% FILE: newpotatoes.tex
%%% DESCRIPTION: input file (tikz code for tripartite graphs)
%%% AUTHOR: williamdemeo@gmail.com
%%% DATE: 2016-03-06
%%%
%%% To insert the code in this file in your own tikzpicture environment, make
%%% sure this file is in your working directory or path and put some lines
%%% like the following in your LaTeX document:
%%%
%%%     \begin{tikzpicture}[scale=0.8]  %% <-- (change the scale if you want)
%%%        %%% FILE: newpotatoes.tex
%%% DESCRIPTION: input file (tikz code for tripartite graphs)
%%% AUTHOR: williamdemeo@gmail.com
%%% DATE: 2016-03-06
%%%
%%% To insert the code in this file in your own tikzpicture environment, make
%%% sure this file is in your working directory or path and put some lines
%%% like the following in your LaTeX document:
%%%
%%%     \begin{tikzpicture}[scale=0.8]  %% <-- (change the scale if you want)
%%%        %%% FILE: newpotatoes.tex
%%% DESCRIPTION: input file (tikz code for tripartite graphs)
%%% AUTHOR: williamdemeo@gmail.com
%%% DATE: 2016-03-06
%%%
%%% To insert the code in this file in your own tikzpicture environment, make
%%% sure this file is in your working directory or path and put some lines
%%% like the following in your LaTeX document:
%%%
%%%     \begin{tikzpicture}[scale=0.8]  %% <-- (change the scale if you want)
%%%        \input{newpotatoes}
%%%     \end{tikzpicture}
%%%
%%% See the file tripartite_graphs.tex for more examples.

%% Specify vertex style here:
\tikzstyle{vstyle}=[circle,draw=blue!50,fill=blue!20,thick,inner sep=0pt,minimum size=3mm]

%% Vertices in North-Western potato:
\foreach \i in {0,1,2,3}
\node[vstyle] (a\i) at (\i,1.25*\i) {\i};

%% Vertices in North-Eastern potato:
\foreach \i in {0,1,2,3}
\node[vstyle] (b\i) at (8.25-\i,1.25*\i) {\i};

%% Vertices in Southern potato:
\foreach \i in {0,1,2,3}
\node[vstyle] (c\i) at (2*\i+1,-2) {\i};

%% ellipse surrounding vertices in NW potato
\draw[dotted,thick] let \p1=(a0), \p2=(a3), \n1={atan2(\y2-\y1,\x2-\x1)}, \n2={veclen(\y2-\y1,\x2-\x1)}
in ($ (a0)!0.5!(a3) $) ellipse [x radius=\n2/2+20pt, y radius=0.7cm,rotate=90-\n1];

%% ellipse surrounding vertices in NE potato
\draw[dotted,thick] let \p1=(b0), \p2=(b3), \n1={atan2(\y2-\y1,\x2-\x1)}, \n2={veclen(\y2-\y1,\x2-\x1)}
in ($ (b0)!0.5!(b3) $) ellipse [x radius=\n2/2+20pt, y radius=0.7cm,rotate=90-\n1];

%% ellipse surrounding vertices in Southern potato
\draw[dotted,thick] let \p1=(c0), \p2=(c3), \n1={atan2(\y2-\y1,\x2-\x1)}, \n2={veclen(\y2-\y1,\x2-\x1)}
in ($ (c0)!0.5!(c3) $) ellipse [x radius=\n2/2+20pt, y radius=0.7cm]; %,rotate=90-\n1];

%%%     \end{tikzpicture}
%%%
%%% See the file tripartite_graphs.tex for more examples.

%% Specify vertex style here:
\tikzstyle{vstyle}=[circle,draw=blue!50,fill=blue!20,thick,inner sep=0pt,minimum size=3mm]

%% Vertices in North-Western potato:
\foreach \i in {0,1,2,3}
\node[vstyle] (a\i) at (\i,1.25*\i) {\i};

%% Vertices in North-Eastern potato:
\foreach \i in {0,1,2,3}
\node[vstyle] (b\i) at (8.25-\i,1.25*\i) {\i};

%% Vertices in Southern potato:
\foreach \i in {0,1,2,3}
\node[vstyle] (c\i) at (2*\i+1,-2) {\i};

%% ellipse surrounding vertices in NW potato
\draw[dotted,thick] let \p1=(a0), \p2=(a3), \n1={atan2(\y2-\y1,\x2-\x1)}, \n2={veclen(\y2-\y1,\x2-\x1)}
in ($ (a0)!0.5!(a3) $) ellipse [x radius=\n2/2+20pt, y radius=0.7cm,rotate=90-\n1];

%% ellipse surrounding vertices in NE potato
\draw[dotted,thick] let \p1=(b0), \p2=(b3), \n1={atan2(\y2-\y1,\x2-\x1)}, \n2={veclen(\y2-\y1,\x2-\x1)}
in ($ (b0)!0.5!(b3) $) ellipse [x radius=\n2/2+20pt, y radius=0.7cm,rotate=90-\n1];

%% ellipse surrounding vertices in Southern potato
\draw[dotted,thick] let \p1=(c0), \p2=(c3), \n1={atan2(\y2-\y1,\x2-\x1)}, \n2={veclen(\y2-\y1,\x2-\x1)}
in ($ (c0)!0.5!(c3) $) ellipse [x radius=\n2/2+20pt, y radius=0.7cm]; %,rotate=90-\n1];

%%%     \end{tikzpicture}
%%%
%%% See the file tripartite_graphs.tex for more examples.

%% Specify vertex style here:
\tikzstyle{vstyle}=[circle,draw=blue!50,fill=blue!20,thick,inner sep=0pt,minimum size=3mm]

%% Vertices in North-Western potato:
\foreach \i in {0,1,2,3}
\node[vstyle] (a\i) at (\i,1.25*\i) {\i};

%% Vertices in North-Eastern potato:
\foreach \i in {0,1,2,3}
\node[vstyle] (b\i) at (8.25-\i,1.25*\i) {\i};

%% Vertices in Southern potato:
\foreach \i in {0,1,2,3}
\node[vstyle] (c\i) at (2*\i+1,-2) {\i};

%% ellipse surrounding vertices in NW potato
\draw[dotted,thick] let \p1=(a0), \p2=(a3), \n1={atan2(\y2-\y1,\x2-\x1)}, \n2={veclen(\y2-\y1,\x2-\x1)}
in ($ (a0)!0.5!(a3) $) ellipse [x radius=\n2/2+20pt, y radius=0.7cm,rotate=90-\n1];

%% ellipse surrounding vertices in NE potato
\draw[dotted,thick] let \p1=(b0), \p2=(b3), \n1={atan2(\y2-\y1,\x2-\x1)}, \n2={veclen(\y2-\y1,\x2-\x1)}
in ($ (b0)!0.5!(b3) $) ellipse [x radius=\n2/2+20pt, y radius=0.7cm,rotate=90-\n1];

%% ellipse surrounding vertices in Southern potato
\draw[dotted,thick] let \p1=(c0), \p2=(c3), \n1={atan2(\y2-\y1,\x2-\x1)}, \n2={veclen(\y2-\y1,\x2-\x1)}
in ($ (c0)!0.5!(c3) $) ellipse [x radius=\n2/2+20pt, y radius=0.7cm]; %,rotate=90-\n1];


      %% Edges: (all edges with labels)
      \begin{scope}[very thick,every node/.style={midway,fill=white}]
        \draw[black]   (a0)--(b0)--(c1)--(a0) node {0};
        \draw[blue]    (a0)--(b1)--(c0)--(a0) node {1};
        \draw[red]     (a1)--(b0)--(c0)--(a1) node {2};
        \draw[mygreen] (a1)--(b1)--(c1)--(a1) node {3};
        \draw[black]   (a0)--(b2)--(c2)--(a0) node {4};
        \draw[blue]    (a0)--(b3)--(c3)--(a0) node {5};
        \draw[red]     (a1)--(b2)--(c3)--(a1) node {6};
        \draw[mygreen] (a1)--(b3)--(c2)--(a1) node {7};
        \draw[black]   (a2)--(b0)--(c2)--(a2) node {8};
        \draw[blue]    (a3)--(b0)--(c3)--(a3) node {9};
        \draw[red]     (a2)--(b1)--(c3)--(a2) node {10};
        \draw[mygreen] (a3)--(b1)--(c2)--(a3) node {11};
        \draw[black]   (a2)--(b2)--(c0)--(a2) node {12};
        \draw[blue]    (a3)--(b3)--(c0)--(a3) node {13};
        \draw[red]     (a2)--(b3)--(c1)--(a2) node {14};
        \draw[mygreen] (a3)--(b2)--(c1)--(a3) node {15};
      \end{scope}

    \end{tikzpicture}

\end{document}
